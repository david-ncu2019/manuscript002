%% 
%% Copyright 2019-2024 Elsevier Ltd
%% 
%% This file is part of the 'CAS Bundle'.
%% --------------------------------------
%% 
%% It may be distributed under the conditions of the LaTeX Project Public
%% License, either version 1.3c of this license or (at your option) any
%% later version.  The latest version of this license is in
%%    http://www.latex-project.org/lppl.txt
%% and version 1.3c or later is part of all distributions of LaTeX
%% version 1999/12/01 or later.
%% 
%% The list of all files belonging to the 'CAS Bundle' is
%% given in the file `manifest.txt'.
%% 
%% Template article for cas-dc documentclass for 
%% double column output.

\documentclass[a4paper,fleqn]{cas-dc}

% --- PACKAGES FROM YOUR ORIGINAL main.tex ---
% Bibliography package (already in template, but confirming options)
\usepackage[authoryear,longnamesfirst]{natbib}

% -- Mathematics --
\usepackage{amsmath}     % Essential for advanced math typesetting.

% -- Figures, Tables, and Captions --
\usepackage{graphicx}    % For including images with \includegraphics
\usepackage{booktabs}    % For professional-quality tables (e.g., \toprule, \midrule)
\usepackage{float}       % Provides the [H] placement option for figures
\usepackage{subcaption}  % For creating subfigures and subtables

% -- Hyperlinks and Cross-referencing --
\usepackage{hyperref}
\hypersetup{
	colorlinks=true,
	linkcolor=blue,
	citecolor=blue,
	urlcolor=blue,
	pdftitle={High-Resolution Subsurface Mapping},
	pdfauthor={Author One et al.},
	pdfkeywords={land subsidence, mitigation, layer-wise compaction},
	pdfdisplaydoctitle=true,
	bookmarksnumbered=true,
	bookmarksopen=true,
	pdfstartview=FitH
}
\usepackage{cleveref} % For smart cross-referencing
\creflabelformat{equation}{#2#1#3}

% -- Utilities --
\usepackage{xcolor}      % For defining and using colors
\usepackage{soul}        % For highlighting text with \hl{}
\usepackage{lineno}      % For line numbering

%%%Author macros
\def\tsc#1{\csdef{#1}{\textsc{\lowercase{#1}}\xspace}}
\tsc{WGM}
\tsc{QE}
%%%

\begin{document}
	\let\WriteBookmarks\relax
	\def\floatpagepagefraction{1}
	\def\textpagefraction{.001}
	
	% Short title
	\shorttitle{High-Resolution Subsurface Mapping for Land Subsidence Mitigation}    
	
	% Short author
	\shortauthors{Author One et al.}  
	
	% Main title of the paper
	\title [mode = title]{High-Resolution Subsurface Mapping Using SBAS-InSAR and Multilayer Compaction Monitoring Wells for Land Subsidence Mitigation}  
	
	% First author
	\author[1]{Author One}
	\cormark[1] % Corresponding author mark
	\ead{author.one@example.com}
	\affiliation[1]{organization={Department of X, University A},
		country={Country}}
	
	% Second author (shares the first affiliation)
	\author[1]{Author Two}
	
	% Third author
	\author[2]{Author Three}
	\ead{author.three@example.com}
	\affiliation[2]{organization={Department of Y, University B},
		country={Country}}
	
	% Fourth author
	\author[3]{Author Four}
	\ead{author.four@example.com}
	\affiliation[3]{organization={Department of Z, University C},
		country={Country}}
	
	% Corresponding author text
	\cortext[1]{Corresponding author}
	
	% Here goes the abstract
	\begin{abstract}
		Land subsidence from excessive groundwater extraction has long affected the Choushui River Fluvial Plain (CRFP) in Taiwan, increasing flood risks and threatening infrastructure like the Taiwan High Speed Rail (THSR). A network of multilayer compaction monitoring wells (MLCWs) aids in understanding subsidence mechanisms, but their high installation costs limit dense deployment. This study integrates SBAS-InSAR surface deformation data with MLCW measurements using geographically weighted regression (GWR) to map layer-wise compaction regionally. This approach enables cost-effective, high-resolution monitoring to inform groundwater management and infrastructure protection.
	\end{abstract}
	
	% Research highlights
	% \begin{highlights}
		% \item Highlight 1
		% \item Highlight 2
		% \item Highlight 3
		% \end{highlights}
	
	% Keywords
	% Each keyword is seperated by \sep
	\begin{keywords}
		land subsidence \sep mitigation \sep layer-wise compaction
	\end{keywords}
	
	\maketitle
	
	% Start line numbering for the main text
	\linenumbers
	
	% Main text
	\section{Introduction}
	\label{sec:intro}
	% Your introduction text goes here, or is loaded from an external file.
	
	\section{Study Area Background}
	\label{sec:studyarea}
	%\section{Study Area Background}
%\label{subsec:study_area}

The CRFP belongs to the western coastal region of central Taiwan. CRFP has an area of approximately 2000 $km^2$, which belongs to the regions of Changhua, Yunlin, and the northern part of the Chiayi counties, with a surface elevation ranging from 0 m to 100 – 150 m above sea level. The CRFP boundary is shaped by Douliu Hill and Bagua Tableland on the eastern side, the Wu River to the north, the Beigang River to the south, and the Taiwan Strait to the west (Figure \ref{fig:studyarea}).
The hydrogeological structures of the study area are divided into three sections from the east to the west, namely proximal fan, middle fan, and distal fan \citep{RN46}. Each section was comprised of various sedimentary materials, with the average grain sizes decreasing from hilly regions to coastal areas. The sedimentary materials in the CRFP are divided into four primary groups, grading from very coarse grains to very fine grains: gravel, coarse sand, fine sand, and clay or silt \citep{RN46}. These materials are weathered products of rock formations located in the upstream watershed of the study area, such as slate, quartzite, shale, sandstone, and mudstone \citep{RN17, RN47}. The borehole profiles suggest that gravel and coarse sand are primarily present at the proximal fan and part of the middle fan, whereas the distal fan mainly witnesses fine-grain materials, including fine sand, clay, and silt (Figure \ref{fig:crosssection}).

% ================================================================================

\begin{figure}[H]
	\centering
	
	% Left large subfigure (A) - takes up left half, full height
	\begin{subfigure}[b]{0.48\textwidth}
		\centering
		\includegraphics[width=\textwidth]{figures/studyarea_dpi150.png}
%		\framebox{\rule{0pt}{10cm}\rule{\textwidth}{0pt}}
%		\caption{}
		\label{fig:studyarea_a}
	\end{subfigure}
	\hfill
	% Right column with three subfigures (B, C, D)
	\begin{minipage}[b]{0.48\textwidth}
		% Top right subfigure (B)
		\begin{subfigure}[b]{\textwidth}
			\centering
			\includegraphics[width=\textwidth]{figures/crosssection.png}
%			\framebox{\rule{0pt}{3cm}\rule{\textwidth}{0pt}}
%			\caption{}
			\label{fig:studyarea_b}
		\end{subfigure}
		
		\vspace{0.1cm}
		
		% Middle right subfigure (C)
		\begin{subfigure}[b]{\textwidth}
			\centering
			\framebox{\rule{0pt}{3cm}\rule{\textwidth}{0pt}}
%			\caption{}
			\label{fig:studyarea_c}
		\end{subfigure}
		
		\vspace{0.1cm}
		
		% Bottom right subfigure (D)
		\begin{subfigure}[b]{\textwidth}
			\centering
			\framebox{\rule{0pt}{3cm}\rule{\textwidth}{0pt}}
%			\caption{}
			\label{fig:studyarea_d}
		\end{subfigure}
	\end{minipage}
	
	\caption[Study Area]{%
		Study Area, Cross Section, and Borehole
		\label{fig:studyarea}
	}	
\end{figure}
%
%\begin{figure}[H]
%	\centering
%	
%	% This is your placeholder for the image.
%	\framebox{\rule{0pt}{7cm}\rule{12.5cm}{0pt}}
%	
%	% Use a short caption in [] for the TOC, and the long one in {} for the figure itself.
%	\caption[Study Area]{% <-- SHORT CAPTION for TOC
%		The location of the study area (bounded by a gray line) and the coverage of Sentinel-1’s SAR images (blue rectangle). The green triangles and black squares stand for GPS stations and leveling benchmarks, respectively. The hydrogeological structures are roughly separated by purple lines.% <-- LONG CAPTION for below the figure
%		\label{fig:studyarea}% The label should be inside the caption.
%	}	
%\end{figure}


% ================================================================================

\begin{figure}[H]
	\centering
%	\includegraphics[width=\textwidth]{figures/crosssection.png}
	\framebox{\rule{0pt}{5cm}\rule{\textwidth}{0pt}}
	\captionsetup{justification=centering}
	\caption[Cross-section]{
		Cross-section showing the distribution of sedimentary materials along profile AA'.
		\label{fig:crosssection}
	}	
\end{figure}
	
	\section{Data Sets}
	\label{sec:dataset}
	% ================================================================================

\subsection{Multilayer Compaction}
\label{subsubsec:mlcw}

A multilayer compaction monitoring well (MLCW) is a specialized borehole extensometer that captures the subtle subsurface compaction by reading measurements at magnetic rings, strategically installed at boundaries between significant aquifers, or transitions between fine and coarse sedimentary materials, as defined by the Geological Survey and Mining Management Agency (GSMMA).  The depth of each MLCW often extends up to 300 m, with 21 to 26 magnetic rings anchored throughout the profile. Based on hydrogeological properties, aquifer units at each well are determined, each containing a number of corresponding magnetic rings, providing measurements of aquifer-specific compaction. The installation and measurement approaches of the MLCWs have been comprehensively described by \citep{Hung2021_MLCW}. In this study, 29 MLCWs were employed, with monthly data collected from April 2016 to November 2021.

% Add these lines to your document's preamble if they're not already there
% \usepackage{siunitx}
% \usepackage{makecell}

\begin{table}[h!]
	% Use \small to reduce the font size just for this table
	\small
	% Reduce the space between columns
	\setlength{\tabcolsep}{2.5pt}
	\centering
	\caption{Monitoring Well Locations, Elevations, and Bottom Depths}
	\label{tab:mlcw_locations_sorted}
	\begin{tabular}{llrrrr}
		\hline
		\textbf{No.} & \textbf{Station} & \textbf{Lon. (\si{\degree})} & \textbf{Lat. (\si{\degree})} & \textbf{Elev. (m)} & \textbf{\makecell{Bottom \\ Depth (m)}} \\
		\hline
		1 & Beichen            & 120.3031           & 23.5760           & 16                     & 320.0                     \\
		2 & Canlin             & 120.2465           & 23.5750           & 9                      & 300.0                     \\
		3 & Dongguang          & 120.2725           & 23.6527           & 10                     & 300.0                     \\
		4 & Erlun              & 120.4155           & 23.7717           & 28                     & 300.0                     \\
		5 & Fengan             & 120.2332           & 23.7892           & 4                      & 300.0                     \\
		6 & Fengrong           & 120.3110           & 23.7907           & 10                     & 300.0                     \\
		7 & Guangfu            & 120.4025           & 23.7414           & 22                     & 300.0                     \\
		8 & Haifeng            & 120.2264           & 23.7643           & 1                      & 200.0                     \\
		9 & Honglun           & 120.3478           & 23.6866           & 17                     & 340.0                     \\
		10 & Hunan            & 120.4791           & 23.9484           & 19                     & 300.0                     \\
		11 & Huwei             & 120.4316           & 23.7153           & 25                     & 300.0                     \\
		12 & Jianyang          & 120.1523           & 23.6341           & 3                      & 200.0                     \\
		13 & Jiaxing           & 120.4596           & 23.6480           & 32                     & 300.0                     \\
		14 & Kecuo             & 120.3343           & 23.6266           & 14                     & 300.0                     \\
		15 & Longyan           & 120.3061           & 23.7227           & 13                     & 300.0                     \\
		16 & Neiliao           & 120.3546           & 23.6077           & 13                     & 300.0                     \\
		17 & Qiaoyi            & 120.4793           & 23.8437           & 34                     & 300.0                     \\
		18 & Tuku              & 120.3898           & 23.6881           & 23                     & 300.0                     \\
		19 & Xigang            & 120.2895           & 23.8603           & 3                      & 300.0                     \\
		20 & Xinjie            & 120.3122           & 23.9025           & 8                      & 300.0                     \\
		21 & Xinsheng          & 120.3943           & 23.9379           & 12                     & 300.0                     \\
		22 & Xinxing           & 120.2223           & 23.7392           & 4                      & 300.0                     \\
		23 & Xinghua           & 120.3947           & 23.8921           & 18                     & 300.0                     \\
		24 & Xiutan            & 120.3496           & 23.6589           & 14                     & 300.0                     \\
		25 & Xizhou            & 120.4981           & 23.8524           & 38                     & 300.0                     \\
		26 & Yiwu              & 120.1953           & 23.5460           & 3                      & 300.0                     \\
		27 & Yuanzhang         & 120.3088           & 23.6533           & 10                     & 300.0                     \\
		28 & Zhennan           & 120.5385           & 23.6986           & 56                     & 300.0                     \\
		29 & Zutang            & 120.4283           & 23.8601           & 27                     & 300.0                     \\
		\hline
	\end{tabular}
\end{table}

\subsection{Cumulative vertical displacement}
\label{subsubsec:vert_disp}

% raw writing first, do not care about structure

This study derived cumulative vertical displacement from an 8-year dataset of Sentinel-1 SAR images \text{(2016 - 2024)}, acquired from both ascending and descending orbits; the image properties are summarized in \Cref{tab:sentinel1_info}. First, the images from each acquisition orbit were processed separately applying the \textbf{\textit{hyp3-isce2}} plugin, part of the Hybrid Pluggable Processing Pipeline (HyP3) \citep{hyp3-isce2}. For interferogram formation, each SAR image was paired with up to four subsequent consecutive images to minimize temporal decorrelation. Next, each image pair was analyzed using the InSAR Scientific Computing Environment 2 (ISCE2) TOPS workflow, which sequentially performs burst-level coregistration of Single Look Complex (SLC) images, interferogram generation, waterbody masking and topographic phase correction with the Copernicus GLO-30 DEM, phase unwrapping via the SNAPHU algorithm, and geocoding to produce unwrapped interferograms suitable for further time-series InSAR analysis. A detailed description of the ISCE2 workflow is available in \citep{isce2_rosen, tops, nesd_tops}.

\begin{table}[H]
	\centering
	\caption{Summary of the Sentinel-1A datasets used in this study.}
	\label{tab:sentinel1_info}
	
	\begin{tabular}{lcc}
		\toprule
		\textbf{Parameters} & \textbf{Ascending} & \textbf{Descending} \\
		\midrule
		Relative Orbit (Path) & 69 & 105 \\
		\multicolumn{1}{l}{Acquisition Period} & \multicolumn{2}{c}{4/2016 – 11/2021} \\
		Number of Images      & 266 & 264 \\
		\multicolumn{1}{l}{Acquisition Mode} & \multicolumn{2}{c}{Interferometric Wide (IW)} \\
		\multicolumn{1}{l}{Polarization} & \multicolumn{2}{c}{VV} \\
		Incidence Angles & 32° – 38° & 38° – 43° \\
		Satellite Headings & 347.63° & 192.37° \\
		\bottomrule
	\end{tabular}
\end{table}

Subsequently, a time-series analysis was conducted on the unwrapped interferograms from ascending and descending orbits separately using the small baseline subset approach, provided by Miami InSAR Time-series software (MintPy) \citep{mintpy_yunjun}. The valid interferogram network was formed through a two-stage selection. The minimum spanning tree (MST) \citep{mst_perissin} first identified the most coherent interferograms to connect all SAR images, by using the inverse of the average spatial coherence of all interferograms as weight. After this, any interferograms not included in the MST were excluded if their average spatial coherence was lower than $\text{0.3}$. After network formation, the optimal values of the interferometric phase timeseries were estimated through a network inversion using the inverse of the phase variance as weight \citep{networkinverse_var}. The temporal coherence, as a product of the network inversion process, was utilized to evaluate the reliability of the estimated value at each pixel \citep{temporal_coherence}. Pixels with temporal coherence ($\gamma_\textit{temp}$) below 0.65 were then masked from further processing. Regarding atmospheric correction, the tropospheric delay components were removed based on the empirical linear relationship between InSAR phase delay and elevation \citep{height_correlation}. All interferograms from both ascending and descending orbits were referenced to a single stable point, which was located far from the subsiding area (\Cref{fig:insar_leveling_a}). Finally, the line-of-sight deformation (${d_\textit{LOS}}$) was decomposed into the east-west (${d_\textit{E-W}}$) and vertical (${d_\textit{U}}$) components, assuming the deformation in north-south direction (${d_\textit{N-S}}$) was negligible. The decomposition employed the incidence angles ($\theta$) and satellite heading angles ($\alpha$) from both ascending ($\textit{asc}$) and descending ($\textit{desc}$) images \citep{hanssen_book}:

\begin{equation}
	\begin{bmatrix}
		d_{\textit{U}} \\
		d_{\textit{E-W}}
	\end{bmatrix}
	=
	\begin{bmatrix}
		\cos(\theta^{\text{asc}}) & \sin(\theta^{\text{asc}}) \cos(\alpha^{\text{asc}}) \\
		\cos(\theta^{\text{desc}}) & \sin(\theta^{\text{desc}}) \cos(\alpha^{\text{desc}})
	\end{bmatrix}^{-1}
	\begin{bmatrix}
		d_{\textit{LOS}}^{\text{asc}} \\
		d_{\textit{LOS}}^{\text{desc}}
	\end{bmatrix}.
	\label{eq:decomposition} %<-- Correct placement and spelling
\end{equation}


Validation of the vertical displacements was performed against precise leveling survey, with benchmarks locations shown in \Cref{fig:studyarea}. First, reliable measurement points ($\gamma_\textit{temp} \ge 0.65$) within a 200 m radius of each benchmark were selected. Average velocities were then derived for these points from both the InSAR and annual leveling survey time series, calculated as the slope of the best-fitting line to their respective displacement time series. Such a comparison based on rates was required due to the different temporal sampling intervals of the two datasets \Cref{fig:insar_leveling}.

\begin{figure}[H]
	\centering
	
	% --- Left Subplot (Figure A) ---
	\begin{subfigure}[b]{0.3\textwidth}
		\centering
		\framebox{\rule{0pt}{8cm}\rule{\textwidth}{0pt}}
		\caption{} % <-- ADD THIS BACK. An empty caption just creates the "(a)".
		\label{fig:insar_leveling_a} % <-- ADD THIS BACK. Give it a unique label.
	\end{subfigure}
	\quad
	% --- Right Subplot (Figure B) ---
	\begin{subfigure}[b]{0.3\textwidth}
		\centering
		\framebox{\rule{0pt}{8cm}\rule{\textwidth}{0pt}}
		\caption{} % <-- ADD THIS BACK. This will create the "(b)".
		\label{fig:insar_leveling_b} % <-- ADD THIS BACK. Give it another unique label.
	\end{subfigure}
	
	% --- Main Caption for the Entire Figure ---
	% You can now refer to (a) and (b) in your main caption if you wish.
	\caption{Comparison of InSAR results and leveling data. (a) The results from InSAR processing. (b) The ground truth data from leveling surveys.}
	\label{fig:insar_leveling} % This label is for the whole figure.
\end{figure}





	
	\section{Methodology}
	\label{sec:method}
	% \section{Methodology}

%\subsection{SBAB-InSAR Processing}
%\label{subsec:sbas}

\subsection{Geographically Temporally Weighted Regression}
\label{subsec:gtwr}

% ================================================================================
% ================================================================================

\subsubsection{Model Calibration}
\label{subsec:gtwr_calib}

% ========================================
% Section: Recap GWR concept
% ========================================

Prior to discussing the GTWR model, it is useful to recap the fundamental concept of the geographically weighted regression (GWR) approach \citep{fotheringham2002geographically}, defined as:

\begin{equation}
y_i = \sum_{k=0}^{p} \beta_k(u_i, v_i) x_{ik} + \varepsilon_i
\label{eq:gwr}
\end{equation}


% ========================================
% Section: Introduce GTWR
% ========================================

\noindent where \(x_{i0} = 1\) for the intercept, \((u_i, v_i)\) are the spatial coordinates of the \(i\)-th point, \(\beta_k(u_i, v_i)\) and \(x_{ik}\) are the coefficient and observed value for the \(k\)-th independent variable, respectively; \(\varepsilon_i\) is the random error, and \(p\) is the number of independent variables (excluding the intercept). \Cref{eq:gwr} indicates that the relationships between dependent and independent variables are spatially nonstationary, which provides a more effective model than the global regression, which assumes spatially constant relationships. However, the environmental quantities not only exhibit spatial but also temporal relationships. Therefore, \citet{huang2010geographically} introduced the GTWR model to account for spatiotemporal nonstationarity. The general form of the GTWR model can be expressed as:

\begin{equation}
y_i = \sum_{k=0}^{p} \beta_k(u_i, v_i, t_i) x_{ik} + \varepsilon_i
\label{eq:gtwr}
\end{equation}

Here, the terms mirror those in GWR, adding the temporal dimension $t_i$. The GTWR model is calibrated based on the assumption that observations closer to the $i$-th point in spatiotemporal coordinate system have greater influence on the estimation of $\beta_k(u_i, v_i, t_i)$ than more distant ones. The estimated parameters $\hat{\boldsymbol{\beta}}(u_i, v_i, t_i)$ can be obtained by:

\begin{equation}
\hat{\boldsymbol{\beta}}(u_i, v_i, t_i) = (\mathbf{X}^T\mathbf{W}(u_i, v_i, t_i)\mathbf{X})^{-1}\mathbf{X}^T\mathbf{W}(u_i, v_i, t_i) \mathbf{y}
\label{eq:coeff_gtwr}
\end{equation}

% =================================================
% emphasize the weight matrix construction
% =================================================
\noindent where $\mathbf{W}(u_i, v_i, t_i)$ is a diagonal matrix with elements denoting the spatiotemporal weights assigned to the observations associated with the $i$-th point. The weights are usually obtained through a kernel function that implement either fixed kernels (using specific distance thresholds) or adaptive kernels (seeking an optimal number of nearest observations), as described by \citet{Paez2002Framework}. Due to the sparse distribution of the MLCW stations across the study area, this study employed the adaptive bi-square kernel function, defined as:

\begin{equation}
W_{ij} = \begin{cases} 
	[1 - (d^{ST}_{ij}/h_i)^2]^2, & \text{if } d^{ST}_{ij} < h_i \\
	0, & \text{otherwise}
\end{cases}
\label{eq:bisquare}
\end{equation}

\noindent where $d^{ST}_{ij}$ denotes the spatiotemporal distances between location $i$ and $j$, and $h_i$ is the adaptive bandwidth, ensuring the same number of observations for each local regression. Since spatial and temporal distances are measured in different units and exhibit different scale effects, \citet{Wu2014Geographically} proposed an integrated formulation, expressed as:

\begin{equation}
\begin{cases}
	d_{ij}^{\text{ST}} = \lambda d_{ij}^{\text{S}} + (1-\lambda) d_{ij}^{\text{T}} + 2 \sqrt{\lambda (1-\lambda) d_{ij}^{\text{S}} d_{ij}^{\text{T}}} \cos(\xi), & t_j < t_i \\
	d_{ij}^{\text{ST}} = \infty, & t_j > t_i
\end{cases}
\label{eq:st_dist}
\end{equation}

\noindent where $t_i$ and $t_j$ are sampled times at locations $i$ and $j$; \( \lambda \text{ and } \xi \in [0, \pi] \) are adjustment parameters. Noticeably, \Cref{eq:st_dist} implies that the GTWR model will simplify to GWR (\cref{eq:gwr}) when $\lambda=1$, and space-time interaction effects reach their maximum when $\xi=0$. In practice, the bandwidths $h_i$, $\lambda$, and $\xi$ can be determined and optimized using a corrected version of the Akaike Information Criterion (AICc) \citep{Hurvich2002_AICc}, defined as:

\begin{equation}
	\text{AICc} = 2n \log_e (\hat{\sigma}) + n \log_e (2\pi) + n \left\{ \frac{n + \text{tr}(\mathbf{S})}{n - 2 - \text{tr}(\mathbf{S})} \right\}
\label{eq:aicc}
\end{equation}

\noindent where \(n\) is the sample size, \(\hat{\sigma}\) is the estimated standard deviation of the error term, and \(\text{tr}(\mathbf{S})\) denotes the trace of the hat matrix $\mathbf{S}$. The hat matrix $\mathbf{S}$ is defined as:

\begin{equation}
	\mathbf{S} = \{S_{ij}\}, \text{ where } S_{ij} = \mathbf{x}_i^T(\mathbf{X}^T \mathbf{W}(u_i, v_i, t_i) \mathbf{X})^{-1}\mathbf{X}^T \mathbf{W}(u_i, v_i, t_i)
	\label{eq:hat_matrix}
\end{equation}

\noindent  where $\mathbf{x}_i^T$ represents the values of independent variables at the $i$-th point. The hat matrix $\mathbf{S}$ transforms observed values into fitted values through spatiotemporal weightings, as in $\hat{\mathbf{y}} = \mathbf{S}\mathbf{y}$, with its trace representing the effective number of parameters in the model \citet{hat_matrix}. The detailed information of these components can be found in \citet{fotheringham2002geographically}.

% In this study, the parameters $\lambda$ and $\xi$ were set to $0.05$ and $0$, respectively, assuming that the temporal effect is dominant and space-time interactive effects are maximized. This configuration constrains the optimization process to focus on determining the optimal bandwidths $h_i$, thereby enhancing the computational efficiency during the calibration procedure.

% ================================================================================
% ================================================================================

\subsubsection{Posterior Uncertainty Assessment}
\label{subsec:gtwr_prediction}

When the optimal bandwidth is obtained through the calibration process, the fitted value $\hat{\mathbf{y}}$ at any unsampled location can be calculated using the hat matrix relationship as previously mentioned. The prediction error variance for each fitted value $\hat{\mathbf{y}}$ is defined as:

\begin{equation}
	\sigma_i^2 = \hat{\sigma}^2 \left(1 + \mathbf{x}_i^T (\mathbf{X}^T \mathbf{W}(u_i, v_i, t_i) \mathbf{X})^{-1} \mathbf{x}_i \right)
	\label{eq:gtwr_variance}
\end{equation}

\noindent where $\hat{\sigma}^2$ represents the overall model error variance, which is defined as:

\begin{equation}
	\hat{\sigma}^2 = \frac{\sum_{i=1}^{N} (y_i - \hat{y}_i)^2}{n - 2\text{tr}(\mathbf{S}) + \text{tr}(\mathbf{S}^T \mathbf{S})}
	\label{eq:model_error_variance}
\end{equation}

The first term in \Cref{eq:gtwr_variance} accounts for the intrinsic model residual variance, while the second term assesses the additional uncertainty that arises when estimating local relationships from limited neighboring observations.


% ================================================================================
% ================================================================================

\subsection{Curve Fitting}
\label{subsec:curve_fit}
	
	\section{Results}
	\label{sec:result}
	% Your results text goes here.
	
	\section{Discussion}
	\label{sec:discuss}
	% Your discussion text goes here.
	
	\section{Conclusion}
	\label{sec:conclusion}
	% Your conclusion text goes here.
	
	% --- Use \appendix command if you have appendices ---
	% \appendix
	% \section{My Appendix}\label{sec:appendix}
	% Appendix content.
	
	% To print the credit authorship contribution details
	% \printcredits % Uncomment if you want to show author contributions
	
	% Acknowledgements
	\section*{Acknowledgements}
	Acknowledge any funding, institutional support, or personal contributions that helped your research.
	
	%% Loading bibliography style file
	\bibliographystyle{cas-model2-names}
	
	% Loading bibliography database
	\bibliography{ref_manu2}
	
\end{document}
