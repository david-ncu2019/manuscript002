%\section{Study Area Background}
%\label{subsec:study_area}

The CRFP belongs to the western coastal region of central Taiwan. CRFP has an area of approximately 2000 $km^2$, which belongs to the regions of Changhua, Yunlin, and the northern part of the Chiayi counties, with a surface elevation ranging from 0 m to 100 – 150 m above sea level. The CRFP boundary is shaped by Douliu Hill and Bagua Tableland on the eastern side, the Wu River to the north, the Beigang River to the south, and the Taiwan Strait to the west (Figure \ref{fig:studyarea}).
The hydrogeological structures of the study area are divided into three sections from the east to the west, namely proximal fan, middle fan, and distal fan \citep{RN46}. Each section was comprised of various sedimentary materials, with the average grain sizes decreasing from hilly regions to coastal areas. The sedimentary materials in the CRFP are divided into four primary groups, grading from very coarse grains to very fine grains: gravel, coarse sand, fine sand, and clay or silt \citep{RN46}. These materials are weathered products of rock formations located in the upstream watershed of the study area, such as slate, quartzite, shale, sandstone, and mudstone \citep{RN17, RN47}. The borehole profiles suggest that gravel and coarse sand are primarily present at the proximal fan and part of the middle fan, whereas the distal fan mainly witnesses fine-grain materials, including fine sand, clay, and silt (Figure \ref{fig:crosssection}).

% ================================================================================

\begin{figure}[H]
	\centering
	
	% Left large subfigure (A) - takes up left half, full height
	\begin{subfigure}[b]{0.48\textwidth}
		\centering
		\includegraphics[width=\textwidth]{figures/studyarea_dpi150.png}
%		\framebox{\rule{0pt}{10cm}\rule{\textwidth}{0pt}}
%		\caption{}
		\label{fig:studyarea_a}
	\end{subfigure}
	\hfill
	% Right column with three subfigures (B, C, D)
	\begin{minipage}[b]{0.48\textwidth}
		% Top right subfigure (B)
		\begin{subfigure}[b]{\textwidth}
			\centering
			\includegraphics[width=\textwidth]{figures/crosssection.png}
%			\framebox{\rule{0pt}{3cm}\rule{\textwidth}{0pt}}
%			\caption{}
			\label{fig:studyarea_b}
		\end{subfigure}
		
		\vspace{0.1cm}
		
		% Middle right subfigure (C)
		\begin{subfigure}[b]{\textwidth}
			\centering
			\framebox{\rule{0pt}{3cm}\rule{\textwidth}{0pt}}
%			\caption{}
			\label{fig:studyarea_c}
		\end{subfigure}
		
		\vspace{0.1cm}
		
		% Bottom right subfigure (D)
		\begin{subfigure}[b]{\textwidth}
			\centering
			\framebox{\rule{0pt}{3cm}\rule{\textwidth}{0pt}}
%			\caption{}
			\label{fig:studyarea_d}
		\end{subfigure}
	\end{minipage}
	
	\caption[Study Area]{%
		Study Area, Cross Section, and Borehole
		\label{fig:studyarea}
	}	
\end{figure}
%
%\begin{figure}[H]
%	\centering
%	
%	% This is your placeholder for the image.
%	\framebox{\rule{0pt}{7cm}\rule{12.5cm}{0pt}}
%	
%	% Use a short caption in [] for the TOC, and the long one in {} for the figure itself.
%	\caption[Study Area]{% <-- SHORT CAPTION for TOC
%		The location of the study area (bounded by a gray line) and the coverage of Sentinel-1’s SAR images (blue rectangle). The green triangles and black squares stand for GPS stations and leveling benchmarks, respectively. The hydrogeological structures are roughly separated by purple lines.% <-- LONG CAPTION for below the figure
%		\label{fig:studyarea}% The label should be inside the caption.
%	}	
%\end{figure}


% ================================================================================

\begin{figure}[H]
	\centering
%	\includegraphics[width=\textwidth]{figures/crosssection.png}
	\framebox{\rule{0pt}{5cm}\rule{\textwidth}{0pt}}
	\captionsetup{justification=centering}
	\caption[Cross-section]{
		Cross-section showing the distribution of sedimentary materials along profile AA'.
		\label{fig:crosssection}
	}	
\end{figure}