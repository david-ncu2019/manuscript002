% \section{Methodology}

\subsection{SBAS-InSAR time series processing}
\label{subsec:sbas}

This study derived cumulative vertical displacement from an 8-year dataset of Sentinel-1 SAR images \text{(2016 - 2024)}, acquired from both ascending and descending orbits; the image properties are summarized in \Cref{tab:sentinel1_info}. First, the images from each acquisition orbit were processed separately applying the \textbf{\textit{hyp3-isce2}} plugin, part of the Hybrid Pluggable Processing Pipeline (HyP3) \citep{hyp3-isce2}. For interferogram formation, each SAR image was paired with up to four subsequent consecutive images to minimize temporal decorrelation. Next, each image pair was analyzed using the InSAR Scientific Computing Environment 2 (ISCE2) TOPS workflow, which sequentially performs burst-level coregistration of Single Look Complex (SLC) images, interferogram generation, waterbody masking and topographic phase correction with the Copernicus GLO-30 DEM, phase unwrapping via the SNAPHU algorithm, and geocoding to produce unwrapped interferograms suitable for further time-series InSAR analysis. A detailed description of the ISCE2 workflow is available in \citep{isce2_rosen, tops, nesd_tops}.

Subsequently, a time-series analysis was conducted on the unwrapped interferograms from ascending and descending orbits separately using the small baseline subset approach, provided by Miami InSAR Time-series software (MintPy) \citep{mintpy_yunjun}. The valid interferogram network was formed through a two-stage selection. The minimum spanning tree (MST) \citep{mst_perissin} first identified the most coherent interferograms to connect all SAR images, by using the inverse of the average spatial coherence of all interferograms as weight. After this, any interferograms not included in the MST were excluded if their average spatial coherence was lower than $\text{0.3}$. After network formation, the optimal values of the interferometric phase timeseries were estimated through a network inversion using the inverse of the phase variance as weight \citep{networkinverse_var}. The temporal coherence, as a product of the network inversion process, was utilized to evaluate the reliability of the estimated value at each pixel \citep{temporal_coherence}. Pixels with temporal coherence ($\gamma_\textit{temp}$) below 0.65 were then masked from further processing. Regarding atmospheric correction, the tropospheric delay components were removed based on the empirical linear relationship between InSAR phase delay and elevation \citep{height_correlation}. All interferograms from both ascending and descending orbits were referenced to a single stable point, which was located far from the subsiding area (\Cref{fig:insar_leveling_a}). Finally, the line-of-sight deformation (${d_\textit{LOS}}$) was decomposed into the east-west (${d_\textit{E-W}}$) and vertical (${d_\textit{U}}$) components, assuming the deformation in north-south direction (${d_\textit{N-S}}$) was negligible. The decomposition employed the incidence angles ($\theta$) and satellite heading angles ($\alpha$) from both ascending ($\textit{asc}$) and descending ($\textit{desc}$) images \citep{hanssen_book}:

\begin{equation}
	\begin{bmatrix}
		d_{\textit{U}} \\
		d_{\textit{E-W}}
	\end{bmatrix}
	=
	\begin{bmatrix}
		\cos(\theta^{\text{asc}}) & \sin(\theta^{\text{asc}}) \cos(\alpha^{\text{asc}}) \\
		\cos(\theta^{\text{desc}}) & \sin(\theta^{\text{desc}}) \cos(\alpha^{\text{desc}})
	\end{bmatrix}^{-1}
	\begin{bmatrix}
		d_{\textit{LOS}}^{\text{asc}} \\
		d_{\textit{LOS}}^{\text{desc}}
	\end{bmatrix}.
	\label{eq:decomposition} %<-- Correct placement and spelling
\end{equation}


Validation of the vertical displacements was performed against precise leveling survey, with benchmarks locations shown in \Cref{fig:studyarea}. First, reliable measurement points ($\gamma_\textit{temp} \ge 0.65$) within a 200 m radius of each benchmark were selected. Average velocities were then derived for these points from both the InSAR and annual leveling survey time series, calculated as the slope of the best-fitting line to their respective displacement time series. Such a comparison based on rates was required due to the different temporal sampling intervals of the two datasets \Cref{fig:insar_leveling}.

\begin{figure}[H]
	\centering
	
	% --- Left Subplot (Figure A) ---
	\begin{subfigure}[b]{0.3\textwidth}
		\centering
		\framebox{\rule{0pt}{8cm}\rule{\textwidth}{0pt}}
		\caption{} % <-- ADD THIS BACK. An empty caption just creates the "(a)".
		\label{fig:insar_leveling_a} % <-- ADD THIS BACK. Give it a unique label.
	\end{subfigure}
	\quad
	% --- Right Subplot (Figure B) ---
	\begin{subfigure}[b]{0.3\textwidth}
		\centering
		\framebox{\rule{0pt}{8cm}\rule{\textwidth}{0pt}}
		\caption{} % <-- ADD THIS BACK. This will create the "(b)".
		\label{fig:insar_leveling_b} % <-- ADD THIS BACK. Give it another unique label.
	\end{subfigure}
	
	% --- Main Caption for the Entire Figure ---
	% You can now refer to (a) and (b) in your main caption if you wish.
	\caption{Comparison of InSAR results and leveling data. (a) The results from InSAR processing. (b) The ground truth data from leveling surveys.}
	\label{fig:insar_leveling} % This label is for the whole figure.
\end{figure}

%\subsection{Geographically Temporally Weighted Regression}
%\label{subsec:gtwr}

% ================================================================================
% ================================================================================

\subsection{Geographically Weighted Regression}
\label{sec:gwr}

This research models the complex relationship between monthly differential displacement from MLCW and InSAR. Traditional regression models, such as Ordinary Least Squares (OLS), are insufficient for this task. OLS assumes that the relationship between the response and predictor variables is constant, or \textbf{spatial stationary}, across the entire study area. This global model is expressed as:
%
\begin{equation}
	y_i = \sum_{k=0}^{p} \beta_k x_{ik} + \varepsilon_i
	\label{eq:ols}
\end{equation}
%
\noindent where \(y_i\) is the dependent variable, \(x_{ik}\) represents the predictors, \(\beta_k\) are the global coefficients, and \(\varepsilon_i\) is the random error at location \(i\). This "one-size-fits-all" assumption, however, fails in hydrogeology. The local geology can dramatically alter the relationship between surface and subsurface deformation from one location to another. Consequently, a global model can misspecify the relationship and produce misleading results \citep{mgwr_book_2017}.

Geographically Weighted Regression (GWR), a local regression technique, addresses this problem of spatial non-stationarity \citep{brunsdon1996, fotheringham2002geographically}. GWR relaxes the stationarity assumption by building a separate, local regression model for every point in the dataset. This approach allows the parameters to vary spatially, fitting a model defined as:
%
\begin{equation}
	y_i = \sum_{k=0}^{p} \beta_k(u_i, v_i) x_{ik} + \varepsilon_i
	\label{eq:gwr}
\end{equation}
%
\noindent The key difference is that the coefficients \(\beta_k\) are no longer global, fixed values, but are specific functions of the geographic coordinates \((u_i, v_i)\) for each location \(i\). In this study, the GWR model is specified to link the monthly differential subsurface compaction directly to the monthly differential surface deformation:
%
\begin{equation}
	\Delta_{\text{MLCW}, n}(i) = \beta_0(u_i, v_i) + \beta_1(u_i, v_i) \cdot \Delta_{\text{InSAR}}(i) + \varepsilon_i
	\label{eq:gwr_mlcw_insar}
\end{equation}
%
\noindent where the dependent variable, \(\Delta_{\text{MLCW}, n}(i)\), is the monthly differential compaction in layer \(n\) at location \(i\), and the independent variable, \(\Delta_{\text{InSAR}}(i)\), is the corresponding InSAR-derived monthly differential surface deformation.

% ================================================================================
% ================================================================================

\subsubsection{Model Calibration}
The GWR model is calibrated by estimating a separate set of coefficients for each location \(i\). This is achieved by assuming that observations closer to \(i\) have a greater influence on the local parameter estimates than more distant ones. The estimated parameters \(\hat{\boldsymbol{\beta}}(u_i, v_i)\) are obtained using a weighted least squares solution:

\begin{equation}
	\hat{\boldsymbol{\beta}}(u_i, v_i) = (\mathbf{X}^T\mathbf{W}(u_i, v_i)\mathbf{X})^{-1}\mathbf{X}^T\mathbf{W}(u_i, v_i) \mathbf{y}
	\label{eq:coeff_gwr}
\end{equation}

\noindent where \(\hat{\boldsymbol{\beta}}(u_i, v_i)\) is the vector of local coefficient estimates, \(\mathbf{X}\) is the matrix of independent variables, \(\mathbf{y}\) is the vector of the dependent variable, and \(\mathbf{W}(u_i, v_i)\) is a diagonal weight matrix specific to location \(i\). The diagonal elements of \(\mathbf{W}(u_i, v_i)\) represent the spatial weights assigned to each observation relative to location \(i\).

These weights are determined by a spatial kernel function and a bandwidth. Due to the sparse distribution of the MLCW stations, this study employed an adaptive bi-square kernel function, defined as:

\begin{equation}
	W_{ij} = \begin{cases} 
		[1 - (d^{S}_{ij}/h_i)^2]^2, & \text{if } d^{S}_{ij} < h_i \\
		0, & \text{otherwise}
	\end{cases}
	\label{eq:bisquare}
\end{equation}

\noindent where \(d^{S}_{ij}\) is the spatial distance between location \(i\) and \(j\). The adaptive bandwidth \(h_i\) is not a fixed distance; instead, it is the distance from \(i\) to its \(k\)-th nearest neighbor. This ensures that each local regression is informed by the same number of observations, regardless of data density. The optimal bandwidth, which is the optimal number of neighbors, is determined by minimizing a corrected version of the Akaike Information Criterion (AICc) \citep{Hurvich2002_AICc}, defined as:

\begin{equation}
	\text{AICc} = 2n \log_e (\hat{\sigma}) + n \log_e (2\pi) + n \left\{ \frac{n + \text{tr}(\mathbf{S})}{n - 2 - \text{tr}(\mathbf{S})} \right\}
	\label{eq:aicc}
\end{equation}

\noindent where \(n\) is the sample size, \(\hat{\sigma}\) is the estimated standard deviation of the error term, and \(\text{tr}(\mathbf{S})\) is the trace of the hat matrix \(\mathbf{S}\).

% ================================================================================
% ================================================================================

\subsubsection{Prediction and Uncertainty Estimation}
Calibrating the local coefficients enables the prediction of the dependent variable \(\hat{y}_i\) at each location \(i\). The model calculates this predicted value as the product of the local independent variables and their corresponding estimated local coefficients:
%
\begin{equation}
	\hat{y}_i = \mathbf{x}_i^T \hat{\boldsymbol{\beta}}(u_i, v_i)
	\label{eq:prediction_gwr}
\end{equation}
%
\noindent where \(\mathbf{x}_i^T\) is the row vector of independent variables at location \(i\). This entire prediction operation can be expressed compactly using the hat matrix \(\mathbf{S}\), which transforms the complete vector of observed values \(\mathbf{y}\) into the vector of predicted values \(\hat{\mathbf{y}}\) (where \(\hat{\mathbf{y}} = \mathbf{S}\mathbf{y}\)). The \(\textit{i}\)-th row of this hat matrix, \(\mathbf{s}_i^T\), defines the precise weighting function used for the prediction at location \(i\):
%
\begin{equation}
	\mathbf{s}_i^T = \mathbf{x}_i^T(\mathbf{X}^T \mathbf{W}(u_i, v_i) \mathbf{X})^{-1}\mathbf{X}^T \mathbf{W}(u_i, v_i)
	\label{eq:hat_matrix_row}
\end{equation}
%
The local regression structure also provides a direct method for quantifying the uncertainty of each prediction. This first requires an estimate of the residual variance, \(\hat{\sigma}^2\), calculated as the residual sum of squares (RSS) divided by the model's effective residual degrees of freedom (RDF):
%
\begin{equation}
	\hat{\sigma}^2 = \frac{\sum_{i=1}^{n} (y_i - \hat{y}_i)^2}{n - 2\text{tr}(\mathbf{S}) + \text{tr}(\mathbf{S}^T\mathbf{S})}
	\label{eq:sigma_squared}
\end{equation}
%
\noindent where \(n\) is the sample size, \((y_i - \hat{y}_i)\) is the residual for observation \(i\), and the denominator represents the effective degrees of freedom for the residuals \citep{fotheringham2002geographically}. The variance of the predicted mean value, \(\text{Var}(\hat{y}_i)\), is estimated from this residual variance and the specific leverage of the prediction point:
%
\begin{equation}
	\text{Var}(\hat{y}_i) = \hat{\sigma}^2 \left( 1 + \underbrace{\mathbf{x}_i^T (\mathbf{X}^T\mathbf{W}(u_i, v_i)\mathbf{X})^{-1} \mathbf{x}_i}_{L_i} \right)
	\label{eq:pred_variance_gwr}
\end{equation}
%
\noindent The term \(L_i\), defined in \Cref{eq:pred_variance_gwr}, represents the leverage of the prediction point \(i\). The standard error of the prediction, \(\text{SE}(\hat{y}_i)\), is the square root of this variance. This standard error is essential, as it allows for the construction of confidence intervals around the predicted \(\hat{y}_i\) values, thereby providing a clear measure of model certainty \citep{fotheringham2002geographically}.