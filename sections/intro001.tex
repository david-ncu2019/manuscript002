%
%
Excessive groundwater withdrawal has been recognized as the primary cause of severe land subsidence in aquifer-dependent regions worldwide, leaving these areas susceptible to infrastructure degradation, inundation, and agricultural production losses \citep{galloway_1999, herrera_2021, jasechko_2024, huning_2024_globalsub, tzampoglou_2023_groundwater}. While approximately 19\% of the global population inhabits areas vulnerable to subsidence, predominantly in coastal plains and alluvial deltas \citep{herrera_2021, nicholls_2021, buffardi_2023_coastalsub}, its magnitude is most significant in Central Valley (California), Mexico City (Mexico), Jakarta (Indonesia), and the Mekong Delta (Vietnam), where recorded cumulative displacements exceed several meters \citep{Miller_2020_california, chaussard_2021_mexico, chaussard_2013_Indonesia, erban_2014_mekong}.

Effective mitigation strategies in subsidence-prone regions depend on the establishment of a comprehensive and reliable monitoring network that not only provides surface deformation patterns but also captures depth-dependent subsurface processes driving them. Common approaches include ground-based geodetic techniques such as spirit leveling and Global Navigation Satellite Systems (GNSS). Both methods provide robust monitoring of land subsidence, but each is accompanied by distinct limitations. Spirit leveling provides highly accurate measurements, but the reliance on labor-intensive surveys typically limits campaigns to annual or semi-annual intervals \citep{poland_1984}. GNSS stations excel at providing high-frequency three-dimensional recordings with millimeter-level precision. However, the establishment of dense GNSS networks is often constrained by prohibitive costs associated with installation and maintenance \citep{bock_2016_GPS}.

% I'm modifying this part to connect with the previous part.
Satellite-based remote sensing, specially Interferometric Synthetic Aperture Radar (InSAR), addresses these spatiotemporal constraints by delivering broad-coverage observations with short to moderate revisit rates \citep{galloway_2007_insar}. For instance, open-access platforms such as Sentinel-1 provide systematic acquisitions every 6 to 12 days. A fundamental limitation persists across these techniques. Measurements quantify only cumulative surface displacement, leaving depth-dependent subsurface compaction distributions unresolved \citep{Galloway_2007}.

Traditional ground-based geodetic approaches offer high measurement precision but suffer from inherent spatial constraints. Continuous GNSS stations deliver millimeter-accuracy vertical displacement time series \citep{Bock_2016}, yet network density remains restricted by significant installation and maintenance costs. Similarly, spirit leveling provides superior vertical accuracy along defined transects \citep{Poland_1984} but entails labor-intensive campaigns typically limited to annual or semi-annual cycles \citep{Galloway_1998}. Satellite-based InSAR addresses these spatial and temporal deficits by generating regional-scale deformation maps with systematic 6-12 day revisit periods via missions such as Sentinel-1 \citep{Osmanoglu_2016}. Nevertheless, a fundamental constraint persists across all three techniques: observations quantify only total cumulative surface displacement, leaving the specific depth-dependent distribution of subsurface compaction unresolved \citep{Galloway_2007}.


%Multilayer compaction monitoring wells (MLCWs) address this limitation by measuring layer-specific vertical compaction at multiple depth intervals using magnetic ring extensometers anchored to discrete stratigraphic boundaries (Riley, 1998). These specialized wells enable direct observation of which aquifer and aquitard layers contribute most to surface subsidence, providing critical information for targeting groundwater management interventions. However, MLCW deployment remains severely constrained by installation costs exceeding US\$100,000 per well and the logistical complexity of drilling 250-300 m boreholes with precise extensometer placement (Sneed \& Brandt, 2007). Consequently, typical MLCW densities remain below 0.02 wells per km² even in intensively monitored regions, yielding sparse point measurements that cannot be directly interpolated across geologically heterogeneous settings where layer contributions vary spatially with sediment compressibility, consolidation history, and pumping patterns (Poland \& Davis, 1969).
%
%Taiwan's Choushui River Fluvial Plain (CRFP) in central western Taiwan exemplifies both the severity of groundwater-induced subsidence and the integration challenges posed by disparate monitoring technologies. Decades of intensive agricultural groundwater extraction have generated cumulative subsidence exceeding 3 m since the 1970s, with ongoing rates reaching 3-7 cm per year in subsidence hotspots threatening critical infrastructure including the Taiwan High Speed Rail (Tung \& Hu, 2012; Hwang et al., 2016). The region hosts one of the world's densest subsidence monitoring networks, including approximately 30 MLCWs measuring four-layer compaction across 0-300 m depth, continuous Sentinel-1 InSAR coverage since 2014, GNSS stations, leveling benchmarks, and daily groundwater level measurements (Hung et al., 2010; Hung et al., 2018). Despite this monitoring density, spatial extrapolation of layer-specific compaction patterns from sparse MLCW locations to the 2,000 km² plain remains unresolved, as standard interpolation methods fail to account for spatial heterogeneity in how subsurface layer compaction couples to surface deformation.
%
%The central challenge is establishing spatially varying relationships between sparse MLCW-measured layer compaction and spatially continuous InSAR surface displacement, not to achieve precise physical decomposition of surface subsidence into exact layer contributions—which requires physics-based consolidation models with constrained hydraulic parameters (Helm, 1975; Teatini et al., 2011)—but rather to identify where and when specific depth intervals exhibit strong versus weak coupling to surface deformation. This study employs Geographically Weighted Regression (GWR) to generate spatially continuous maps of layer-specific subsidence sensitivity across the CRFP. GWR estimates location-specific regression coefficients by fitting local models using distance-weighted neighboring observations (Brunsdon et al., 1996; Fotheringham et al., 2002), accommodating spatial non-stationarity in layer-surface relationships. Critically, GWR coefficients are interpreted as sensitivity indices rather than exact physical contributions, where larger values indicate locations at which a given layer's compaction more strongly associates with surface subsidence. Analysis integrates 67 months of SBAS-InSAR and four-layer MLCW data spanning 2015-2021, with synthetic validation experiments and correlation analyses testing whether spatial patterns reflect physical controls or statistical artifacts.
%
%The methodology enables identification of which depth intervals dominate subsidence contributions, where coupling strength varies spatially, and how patterns relate to geological compressibility and hydrological forcing. For linear infrastructure such as the Taiwan High Speed Rail, spatial sensitivity maps identify differential subsidence risk zones where layer-specific compaction patterns threaten track alignment and structural integrity. Section 2 describes the study area hydrogeology and datasets. Section 3 presents GWR methodology, synthetic validation experiments, and correlation analysis frameworks. Section 4 reports sensitivity mapping results and validation findings. Sections 5 and 6 discuss implications for subsidence management, acknowledge limitations, and outline future research directions.